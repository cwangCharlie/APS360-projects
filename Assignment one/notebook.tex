
% Default to the notebook output style

    


% Inherit from the specified cell style.




    
\documentclass[11pt]{article}

    
    
    \usepackage[T1]{fontenc}
    % Nicer default font (+ math font) than Computer Modern for most use cases
    \usepackage{mathpazo}

    % Basic figure setup, for now with no caption control since it's done
    % automatically by Pandoc (which extracts ![](path) syntax from Markdown).
    \usepackage{graphicx}
    % We will generate all images so they have a width \maxwidth. This means
    % that they will get their normal width if they fit onto the page, but
    % are scaled down if they would overflow the margins.
    \makeatletter
    \def\maxwidth{\ifdim\Gin@nat@width>\linewidth\linewidth
    \else\Gin@nat@width\fi}
    \makeatother
    \let\Oldincludegraphics\includegraphics
    % Set max figure width to be 80% of text width, for now hardcoded.
    \renewcommand{\includegraphics}[1]{\Oldincludegraphics[width=.8\maxwidth]{#1}}
    % Ensure that by default, figures have no caption (until we provide a
    % proper Figure object with a Caption API and a way to capture that
    % in the conversion process - todo).
    \usepackage{caption}
    \DeclareCaptionLabelFormat{nolabel}{}
    \captionsetup{labelformat=nolabel}

    \usepackage{adjustbox} % Used to constrain images to a maximum size 
    \usepackage{xcolor} % Allow colors to be defined
    \usepackage{enumerate} % Needed for markdown enumerations to work
    \usepackage{geometry} % Used to adjust the document margins
    \usepackage{amsmath} % Equations
    \usepackage{amssymb} % Equations
    \usepackage{textcomp} % defines textquotesingle
    % Hack from http://tex.stackexchange.com/a/47451/13684:
    \AtBeginDocument{%
        \def\PYZsq{\textquotesingle}% Upright quotes in Pygmentized code
    }
    \usepackage{upquote} % Upright quotes for verbatim code
    \usepackage{eurosym} % defines \euro
    \usepackage[mathletters]{ucs} % Extended unicode (utf-8) support
    \usepackage[utf8x]{inputenc} % Allow utf-8 characters in the tex document
    \usepackage{fancyvrb} % verbatim replacement that allows latex
    \usepackage{grffile} % extends the file name processing of package graphics 
                         % to support a larger range 
    % The hyperref package gives us a pdf with properly built
    % internal navigation ('pdf bookmarks' for the table of contents,
    % internal cross-reference links, web links for URLs, etc.)
    \usepackage{hyperref}
    \usepackage{longtable} % longtable support required by pandoc >1.10
    \usepackage{booktabs}  % table support for pandoc > 1.12.2
    \usepackage[inline]{enumitem} % IRkernel/repr support (it uses the enumerate* environment)
    \usepackage[normalem]{ulem} % ulem is needed to support strikethroughs (\sout)
                                % normalem makes italics be italics, not underlines
    

    
    
    % Colors for the hyperref package
    \definecolor{urlcolor}{rgb}{0,.145,.698}
    \definecolor{linkcolor}{rgb}{.71,0.21,0.01}
    \definecolor{citecolor}{rgb}{.12,.54,.11}

    % ANSI colors
    \definecolor{ansi-black}{HTML}{3E424D}
    \definecolor{ansi-black-intense}{HTML}{282C36}
    \definecolor{ansi-red}{HTML}{E75C58}
    \definecolor{ansi-red-intense}{HTML}{B22B31}
    \definecolor{ansi-green}{HTML}{00A250}
    \definecolor{ansi-green-intense}{HTML}{007427}
    \definecolor{ansi-yellow}{HTML}{DDB62B}
    \definecolor{ansi-yellow-intense}{HTML}{B27D12}
    \definecolor{ansi-blue}{HTML}{208FFB}
    \definecolor{ansi-blue-intense}{HTML}{0065CA}
    \definecolor{ansi-magenta}{HTML}{D160C4}
    \definecolor{ansi-magenta-intense}{HTML}{A03196}
    \definecolor{ansi-cyan}{HTML}{60C6C8}
    \definecolor{ansi-cyan-intense}{HTML}{258F8F}
    \definecolor{ansi-white}{HTML}{C5C1B4}
    \definecolor{ansi-white-intense}{HTML}{A1A6B2}

    % commands and environments needed by pandoc snippets
    % extracted from the output of `pandoc -s`
    \providecommand{\tightlist}{%
      \setlength{\itemsep}{0pt}\setlength{\parskip}{0pt}}
    \DefineVerbatimEnvironment{Highlighting}{Verbatim}{commandchars=\\\{\}}
    % Add ',fontsize=\small' for more characters per line
    \newenvironment{Shaded}{}{}
    \newcommand{\KeywordTok}[1]{\textcolor[rgb]{0.00,0.44,0.13}{\textbf{{#1}}}}
    \newcommand{\DataTypeTok}[1]{\textcolor[rgb]{0.56,0.13,0.00}{{#1}}}
    \newcommand{\DecValTok}[1]{\textcolor[rgb]{0.25,0.63,0.44}{{#1}}}
    \newcommand{\BaseNTok}[1]{\textcolor[rgb]{0.25,0.63,0.44}{{#1}}}
    \newcommand{\FloatTok}[1]{\textcolor[rgb]{0.25,0.63,0.44}{{#1}}}
    \newcommand{\CharTok}[1]{\textcolor[rgb]{0.25,0.44,0.63}{{#1}}}
    \newcommand{\StringTok}[1]{\textcolor[rgb]{0.25,0.44,0.63}{{#1}}}
    \newcommand{\CommentTok}[1]{\textcolor[rgb]{0.38,0.63,0.69}{\textit{{#1}}}}
    \newcommand{\OtherTok}[1]{\textcolor[rgb]{0.00,0.44,0.13}{{#1}}}
    \newcommand{\AlertTok}[1]{\textcolor[rgb]{1.00,0.00,0.00}{\textbf{{#1}}}}
    \newcommand{\FunctionTok}[1]{\textcolor[rgb]{0.02,0.16,0.49}{{#1}}}
    \newcommand{\RegionMarkerTok}[1]{{#1}}
    \newcommand{\ErrorTok}[1]{\textcolor[rgb]{1.00,0.00,0.00}{\textbf{{#1}}}}
    \newcommand{\NormalTok}[1]{{#1}}
    
    % Additional commands for more recent versions of Pandoc
    \newcommand{\ConstantTok}[1]{\textcolor[rgb]{0.53,0.00,0.00}{{#1}}}
    \newcommand{\SpecialCharTok}[1]{\textcolor[rgb]{0.25,0.44,0.63}{{#1}}}
    \newcommand{\VerbatimStringTok}[1]{\textcolor[rgb]{0.25,0.44,0.63}{{#1}}}
    \newcommand{\SpecialStringTok}[1]{\textcolor[rgb]{0.73,0.40,0.53}{{#1}}}
    \newcommand{\ImportTok}[1]{{#1}}
    \newcommand{\DocumentationTok}[1]{\textcolor[rgb]{0.73,0.13,0.13}{\textit{{#1}}}}
    \newcommand{\AnnotationTok}[1]{\textcolor[rgb]{0.38,0.63,0.69}{\textbf{\textit{{#1}}}}}
    \newcommand{\CommentVarTok}[1]{\textcolor[rgb]{0.38,0.63,0.69}{\textbf{\textit{{#1}}}}}
    \newcommand{\VariableTok}[1]{\textcolor[rgb]{0.10,0.09,0.49}{{#1}}}
    \newcommand{\ControlFlowTok}[1]{\textcolor[rgb]{0.00,0.44,0.13}{\textbf{{#1}}}}
    \newcommand{\OperatorTok}[1]{\textcolor[rgb]{0.40,0.40,0.40}{{#1}}}
    \newcommand{\BuiltInTok}[1]{{#1}}
    \newcommand{\ExtensionTok}[1]{{#1}}
    \newcommand{\PreprocessorTok}[1]{\textcolor[rgb]{0.74,0.48,0.00}{{#1}}}
    \newcommand{\AttributeTok}[1]{\textcolor[rgb]{0.49,0.56,0.16}{{#1}}}
    \newcommand{\InformationTok}[1]{\textcolor[rgb]{0.38,0.63,0.69}{\textbf{\textit{{#1}}}}}
    \newcommand{\WarningTok}[1]{\textcolor[rgb]{0.38,0.63,0.69}{\textbf{\textit{{#1}}}}}
    
    
    % Define a nice break command that doesn't care if a line doesn't already
    % exist.
    \def\br{\hspace*{\fill} \\* }
    % Math Jax compatability definitions
    \def\gt{>}
    \def\lt{<}
    % Document parameters
    \title{a1}
    
    
    

    % Pygments definitions
    
\makeatletter
\def\PY@reset{\let\PY@it=\relax \let\PY@bf=\relax%
    \let\PY@ul=\relax \let\PY@tc=\relax%
    \let\PY@bc=\relax \let\PY@ff=\relax}
\def\PY@tok#1{\csname PY@tok@#1\endcsname}
\def\PY@toks#1+{\ifx\relax#1\empty\else%
    \PY@tok{#1}\expandafter\PY@toks\fi}
\def\PY@do#1{\PY@bc{\PY@tc{\PY@ul{%
    \PY@it{\PY@bf{\PY@ff{#1}}}}}}}
\def\PY#1#2{\PY@reset\PY@toks#1+\relax+\PY@do{#2}}

\expandafter\def\csname PY@tok@w\endcsname{\def\PY@tc##1{\textcolor[rgb]{0.73,0.73,0.73}{##1}}}
\expandafter\def\csname PY@tok@c\endcsname{\let\PY@it=\textit\def\PY@tc##1{\textcolor[rgb]{0.25,0.50,0.50}{##1}}}
\expandafter\def\csname PY@tok@cp\endcsname{\def\PY@tc##1{\textcolor[rgb]{0.74,0.48,0.00}{##1}}}
\expandafter\def\csname PY@tok@k\endcsname{\let\PY@bf=\textbf\def\PY@tc##1{\textcolor[rgb]{0.00,0.50,0.00}{##1}}}
\expandafter\def\csname PY@tok@kp\endcsname{\def\PY@tc##1{\textcolor[rgb]{0.00,0.50,0.00}{##1}}}
\expandafter\def\csname PY@tok@kt\endcsname{\def\PY@tc##1{\textcolor[rgb]{0.69,0.00,0.25}{##1}}}
\expandafter\def\csname PY@tok@o\endcsname{\def\PY@tc##1{\textcolor[rgb]{0.40,0.40,0.40}{##1}}}
\expandafter\def\csname PY@tok@ow\endcsname{\let\PY@bf=\textbf\def\PY@tc##1{\textcolor[rgb]{0.67,0.13,1.00}{##1}}}
\expandafter\def\csname PY@tok@nb\endcsname{\def\PY@tc##1{\textcolor[rgb]{0.00,0.50,0.00}{##1}}}
\expandafter\def\csname PY@tok@nf\endcsname{\def\PY@tc##1{\textcolor[rgb]{0.00,0.00,1.00}{##1}}}
\expandafter\def\csname PY@tok@nc\endcsname{\let\PY@bf=\textbf\def\PY@tc##1{\textcolor[rgb]{0.00,0.00,1.00}{##1}}}
\expandafter\def\csname PY@tok@nn\endcsname{\let\PY@bf=\textbf\def\PY@tc##1{\textcolor[rgb]{0.00,0.00,1.00}{##1}}}
\expandafter\def\csname PY@tok@ne\endcsname{\let\PY@bf=\textbf\def\PY@tc##1{\textcolor[rgb]{0.82,0.25,0.23}{##1}}}
\expandafter\def\csname PY@tok@nv\endcsname{\def\PY@tc##1{\textcolor[rgb]{0.10,0.09,0.49}{##1}}}
\expandafter\def\csname PY@tok@no\endcsname{\def\PY@tc##1{\textcolor[rgb]{0.53,0.00,0.00}{##1}}}
\expandafter\def\csname PY@tok@nl\endcsname{\def\PY@tc##1{\textcolor[rgb]{0.63,0.63,0.00}{##1}}}
\expandafter\def\csname PY@tok@ni\endcsname{\let\PY@bf=\textbf\def\PY@tc##1{\textcolor[rgb]{0.60,0.60,0.60}{##1}}}
\expandafter\def\csname PY@tok@na\endcsname{\def\PY@tc##1{\textcolor[rgb]{0.49,0.56,0.16}{##1}}}
\expandafter\def\csname PY@tok@nt\endcsname{\let\PY@bf=\textbf\def\PY@tc##1{\textcolor[rgb]{0.00,0.50,0.00}{##1}}}
\expandafter\def\csname PY@tok@nd\endcsname{\def\PY@tc##1{\textcolor[rgb]{0.67,0.13,1.00}{##1}}}
\expandafter\def\csname PY@tok@s\endcsname{\def\PY@tc##1{\textcolor[rgb]{0.73,0.13,0.13}{##1}}}
\expandafter\def\csname PY@tok@sd\endcsname{\let\PY@it=\textit\def\PY@tc##1{\textcolor[rgb]{0.73,0.13,0.13}{##1}}}
\expandafter\def\csname PY@tok@si\endcsname{\let\PY@bf=\textbf\def\PY@tc##1{\textcolor[rgb]{0.73,0.40,0.53}{##1}}}
\expandafter\def\csname PY@tok@se\endcsname{\let\PY@bf=\textbf\def\PY@tc##1{\textcolor[rgb]{0.73,0.40,0.13}{##1}}}
\expandafter\def\csname PY@tok@sr\endcsname{\def\PY@tc##1{\textcolor[rgb]{0.73,0.40,0.53}{##1}}}
\expandafter\def\csname PY@tok@ss\endcsname{\def\PY@tc##1{\textcolor[rgb]{0.10,0.09,0.49}{##1}}}
\expandafter\def\csname PY@tok@sx\endcsname{\def\PY@tc##1{\textcolor[rgb]{0.00,0.50,0.00}{##1}}}
\expandafter\def\csname PY@tok@m\endcsname{\def\PY@tc##1{\textcolor[rgb]{0.40,0.40,0.40}{##1}}}
\expandafter\def\csname PY@tok@gh\endcsname{\let\PY@bf=\textbf\def\PY@tc##1{\textcolor[rgb]{0.00,0.00,0.50}{##1}}}
\expandafter\def\csname PY@tok@gu\endcsname{\let\PY@bf=\textbf\def\PY@tc##1{\textcolor[rgb]{0.50,0.00,0.50}{##1}}}
\expandafter\def\csname PY@tok@gd\endcsname{\def\PY@tc##1{\textcolor[rgb]{0.63,0.00,0.00}{##1}}}
\expandafter\def\csname PY@tok@gi\endcsname{\def\PY@tc##1{\textcolor[rgb]{0.00,0.63,0.00}{##1}}}
\expandafter\def\csname PY@tok@gr\endcsname{\def\PY@tc##1{\textcolor[rgb]{1.00,0.00,0.00}{##1}}}
\expandafter\def\csname PY@tok@ge\endcsname{\let\PY@it=\textit}
\expandafter\def\csname PY@tok@gs\endcsname{\let\PY@bf=\textbf}
\expandafter\def\csname PY@tok@gp\endcsname{\let\PY@bf=\textbf\def\PY@tc##1{\textcolor[rgb]{0.00,0.00,0.50}{##1}}}
\expandafter\def\csname PY@tok@go\endcsname{\def\PY@tc##1{\textcolor[rgb]{0.53,0.53,0.53}{##1}}}
\expandafter\def\csname PY@tok@gt\endcsname{\def\PY@tc##1{\textcolor[rgb]{0.00,0.27,0.87}{##1}}}
\expandafter\def\csname PY@tok@err\endcsname{\def\PY@bc##1{\setlength{\fboxsep}{0pt}\fcolorbox[rgb]{1.00,0.00,0.00}{1,1,1}{\strut ##1}}}
\expandafter\def\csname PY@tok@kc\endcsname{\let\PY@bf=\textbf\def\PY@tc##1{\textcolor[rgb]{0.00,0.50,0.00}{##1}}}
\expandafter\def\csname PY@tok@kd\endcsname{\let\PY@bf=\textbf\def\PY@tc##1{\textcolor[rgb]{0.00,0.50,0.00}{##1}}}
\expandafter\def\csname PY@tok@kn\endcsname{\let\PY@bf=\textbf\def\PY@tc##1{\textcolor[rgb]{0.00,0.50,0.00}{##1}}}
\expandafter\def\csname PY@tok@kr\endcsname{\let\PY@bf=\textbf\def\PY@tc##1{\textcolor[rgb]{0.00,0.50,0.00}{##1}}}
\expandafter\def\csname PY@tok@bp\endcsname{\def\PY@tc##1{\textcolor[rgb]{0.00,0.50,0.00}{##1}}}
\expandafter\def\csname PY@tok@fm\endcsname{\def\PY@tc##1{\textcolor[rgb]{0.00,0.00,1.00}{##1}}}
\expandafter\def\csname PY@tok@vc\endcsname{\def\PY@tc##1{\textcolor[rgb]{0.10,0.09,0.49}{##1}}}
\expandafter\def\csname PY@tok@vg\endcsname{\def\PY@tc##1{\textcolor[rgb]{0.10,0.09,0.49}{##1}}}
\expandafter\def\csname PY@tok@vi\endcsname{\def\PY@tc##1{\textcolor[rgb]{0.10,0.09,0.49}{##1}}}
\expandafter\def\csname PY@tok@vm\endcsname{\def\PY@tc##1{\textcolor[rgb]{0.10,0.09,0.49}{##1}}}
\expandafter\def\csname PY@tok@sa\endcsname{\def\PY@tc##1{\textcolor[rgb]{0.73,0.13,0.13}{##1}}}
\expandafter\def\csname PY@tok@sb\endcsname{\def\PY@tc##1{\textcolor[rgb]{0.73,0.13,0.13}{##1}}}
\expandafter\def\csname PY@tok@sc\endcsname{\def\PY@tc##1{\textcolor[rgb]{0.73,0.13,0.13}{##1}}}
\expandafter\def\csname PY@tok@dl\endcsname{\def\PY@tc##1{\textcolor[rgb]{0.73,0.13,0.13}{##1}}}
\expandafter\def\csname PY@tok@s2\endcsname{\def\PY@tc##1{\textcolor[rgb]{0.73,0.13,0.13}{##1}}}
\expandafter\def\csname PY@tok@sh\endcsname{\def\PY@tc##1{\textcolor[rgb]{0.73,0.13,0.13}{##1}}}
\expandafter\def\csname PY@tok@s1\endcsname{\def\PY@tc##1{\textcolor[rgb]{0.73,0.13,0.13}{##1}}}
\expandafter\def\csname PY@tok@mb\endcsname{\def\PY@tc##1{\textcolor[rgb]{0.40,0.40,0.40}{##1}}}
\expandafter\def\csname PY@tok@mf\endcsname{\def\PY@tc##1{\textcolor[rgb]{0.40,0.40,0.40}{##1}}}
\expandafter\def\csname PY@tok@mh\endcsname{\def\PY@tc##1{\textcolor[rgb]{0.40,0.40,0.40}{##1}}}
\expandafter\def\csname PY@tok@mi\endcsname{\def\PY@tc##1{\textcolor[rgb]{0.40,0.40,0.40}{##1}}}
\expandafter\def\csname PY@tok@il\endcsname{\def\PY@tc##1{\textcolor[rgb]{0.40,0.40,0.40}{##1}}}
\expandafter\def\csname PY@tok@mo\endcsname{\def\PY@tc##1{\textcolor[rgb]{0.40,0.40,0.40}{##1}}}
\expandafter\def\csname PY@tok@ch\endcsname{\let\PY@it=\textit\def\PY@tc##1{\textcolor[rgb]{0.25,0.50,0.50}{##1}}}
\expandafter\def\csname PY@tok@cm\endcsname{\let\PY@it=\textit\def\PY@tc##1{\textcolor[rgb]{0.25,0.50,0.50}{##1}}}
\expandafter\def\csname PY@tok@cpf\endcsname{\let\PY@it=\textit\def\PY@tc##1{\textcolor[rgb]{0.25,0.50,0.50}{##1}}}
\expandafter\def\csname PY@tok@c1\endcsname{\let\PY@it=\textit\def\PY@tc##1{\textcolor[rgb]{0.25,0.50,0.50}{##1}}}
\expandafter\def\csname PY@tok@cs\endcsname{\let\PY@it=\textit\def\PY@tc##1{\textcolor[rgb]{0.25,0.50,0.50}{##1}}}

\def\PYZbs{\char`\\}
\def\PYZus{\char`\_}
\def\PYZob{\char`\{}
\def\PYZcb{\char`\}}
\def\PYZca{\char`\^}
\def\PYZam{\char`\&}
\def\PYZlt{\char`\<}
\def\PYZgt{\char`\>}
\def\PYZsh{\char`\#}
\def\PYZpc{\char`\%}
\def\PYZdl{\char`\$}
\def\PYZhy{\char`\-}
\def\PYZsq{\char`\'}
\def\PYZdq{\char`\"}
\def\PYZti{\char`\~}
% for compatibility with earlier versions
\def\PYZat{@}
\def\PYZlb{[}
\def\PYZrb{]}
\makeatother


    % Exact colors from NB
    \definecolor{incolor}{rgb}{0.0, 0.0, 0.5}
    \definecolor{outcolor}{rgb}{0.545, 0.0, 0.0}



    
    % Prevent overflowing lines due to hard-to-break entities
    \sloppy 
    % Setup hyperref package
    \hypersetup{
      breaklinks=true,  % so long urls are correctly broken across lines
      colorlinks=true,
      urlcolor=urlcolor,
      linkcolor=linkcolor,
      citecolor=citecolor,
      }
    % Slightly bigger margins than the latex defaults
    
    \geometry{verbose,tmargin=1in,bmargin=1in,lmargin=1in,rmargin=1in}
    
    

    \begin{document}
    
    
    \maketitle
    
    

    
    \section{Assignment 1. Python and
libraries}\label{assignment-1.-python-and-libraries}

\textbf{Deadline}: January 20, 9pm.

\textbf{Late Penalty}: See Syllabus

\textbf{TAs}: Andrew Jung

Welcome to the first assignment of APS360! This assignment is a warm up
to get you used to the programming environment used in the course, and
also to help you review and renew your knowledge of Python and relevant
Python libraries. The assignment must be done individually. Please
recall that the University of Toronto plagarism rules apply.

By the end of this assignment, you should be able to:

\begin{enumerate}
\def\labelenumi{\arabic{enumi}.}
\tightlist
\item
  Set up the computing environment used in this course: the Python
  language interpreter, Jupyter Notebook, and the PyCharm Integraded
  Development Environment (IDE)
\item
  Write basic, object-oriented Python code.
\item
  Be able to perform matrix operations using \texttt{numpy}.
\item
  Be able to plot using \texttt{matplotlib}.
\item
  Be able to load, process, and visualize image data.
\end{enumerate}

\subsubsection{What to submit}\label{what-to-submit}

Submit a PDF file containing all your code, outputs, and write-up from
parts 1-4. \textbf{Do not submit any other files produced by your code.}

Completing this assignment using Jupyter Notebook is highly recommended
(though not necessarily for all subsequent assignments). If you are
using Jupyter Notebook, you can export a PDF file using the menu option
\texttt{File\ -\textgreater{}\ Download\ As\ -\textgreater{}\ PDF\ via\ LaTeX\ (pdf)}

    \subsection{Part 0. Environment Setup;
Readings}\label{part-0.-environment-setup-readings}

Your first step is to set up your development environment. We'll be
using the \textbf{Anaconda} distribution of Python 3.6. Following the
instructions to install Python3.6, PyCharm, Jupyter Notebook:

https://www.cs.toronto.edu/\textasciitilde{}lczhang/360/files/install.pdf

To prepare for the rest of the assignment, the preparatory readings in
section 2 of the installation instructions are very helpful.

    \subsection{Part 1. Python Basics {[}6
pt{]}}\label{part-1.-python-basics-6-pt}

The purpose of this section is to get you used to the basics of Python,
including working with functions, numbers, lists, and strings.

    \subsubsection{Part (a) -\/- 3pt}\label{part-a----3pt}

Write a function \texttt{sum\_of\_squares} that computes the sum of
squares up to \texttt{n}.

    \begin{Verbatim}[commandchars=\\\{\}]
{\color{incolor}In [{\color{incolor}14}]:} \PY{k}{def} \PY{n+nf}{sum\PYZus{}of\PYZus{}squares}\PY{p}{(}\PY{n}{n}\PY{p}{)}\PY{p}{:}
             \PY{n+nb}{sum} \PY{o}{=} \PY{l+m+mi}{0}
             \PY{k}{for} \PY{n}{i} \PY{o+ow}{in} \PY{n+nb}{range} \PY{p}{(}\PY{l+m+mi}{1}\PY{p}{,}\PY{n}{n}\PY{o}{+}\PY{l+m+mi}{1}\PY{p}{)}\PY{p}{:}
                 \PY{n+nb}{sum} \PY{o}{=} \PY{n}{i}\PY{o}{*}\PY{n}{i} \PY{o}{+} \PY{n+nb}{sum} 
             
             \PY{k}{return} \PY{n+nb}{sum}
             \PY{l+s+sd}{\PYZdq{}\PYZdq{}\PYZdq{}Return the sum (1 + 2\PYZca{}2 + 3\PYZca{}2 + ... + n\PYZca{}2)}
         \PY{l+s+sd}{    }
         \PY{l+s+sd}{    Precondition: n \PYZgt{} 0, type(n) == int}
         \PY{l+s+sd}{    }
         \PY{l+s+sd}{    \PYZgt{}\PYZgt{}\PYZgt{} sum\PYZus{}of\PYZus{}squares(3)}
         \PY{l+s+sd}{    14}
         \PY{l+s+sd}{    \PYZgt{}\PYZgt{}\PYZgt{} sum\PYZus{}of\PYZus{}squares(1)}
         \PY{l+s+sd}{    1}
         \PY{l+s+sd}{    \PYZdq{}\PYZdq{}\PYZdq{}}
             
         \PY{n}{sum\PYZus{}of\PYZus{}squares}\PY{p}{(}\PY{l+m+mi}{3}\PY{p}{)}
\end{Verbatim}


\begin{Verbatim}[commandchars=\\\{\}]
{\color{outcolor}Out[{\color{outcolor}14}]:} 14
\end{Verbatim}
            
    \subsubsection{Part (b) -\/- 3pt}\label{part-b----3pt}

Write a function \texttt{word\_lengths} that takes a sentence (string),
computes the length of each word in that sentence, and returns the
length of each word in a list. You can assume that words are always
separated by a space character \texttt{"\ "}.

Hint: recall the \texttt{str.split} function in Python. If you arenot
sure how this function works, try typing \texttt{help(str.split)} into a
Python shell, or checkout
https://docs.python.org/3.6/library/stdtypes.html\#str.split

    \begin{Verbatim}[commandchars=\\\{\}]
{\color{incolor}In [{\color{incolor}4}]:} \PY{n}{help}\PY{p}{(}\PY{n+nb}{str}\PY{o}{.}\PY{n}{split}\PY{p}{)}
\end{Verbatim}


    \begin{Verbatim}[commandchars=\\\{\}]
Help on method\_descriptor:

split({\ldots})
    S.split(sep=None, maxsplit=-1) -> list of strings
    
    Return a list of the words in S, using sep as the
    delimiter string.  If maxsplit is given, at most maxsplit
    splits are done. If sep is not specified or is None, any
    whitespace string is a separator and empty strings are
    removed from the result.


    \end{Verbatim}

    \begin{Verbatim}[commandchars=\\\{\}]
{\color{incolor}In [{\color{incolor}8}]:} \PY{k}{def} \PY{n+nf}{word\PYZus{}lengths}\PY{p}{(}\PY{n}{sentence}\PY{p}{)}\PY{p}{:}
            \PY{n}{words}\PY{o}{=} \PY{n}{sentence}\PY{o}{.}\PY{n}{split}\PY{p}{(}\PY{l+s+s2}{\PYZdq{}}\PY{l+s+s2}{ }\PY{l+s+s2}{\PYZdq{}}\PY{p}{)}
            \PY{n}{result} \PY{o}{=} \PY{p}{[}\PY{p}{]}
            \PY{k}{for} \PY{n}{i} \PY{o+ow}{in} \PY{n}{words}\PY{p}{:}
                \PY{n}{result}\PY{o}{.}\PY{n}{append}\PY{p}{(}\PY{n+nb}{len}\PY{p}{(}\PY{n}{i}\PY{p}{)}\PY{p}{)}
            \PY{k}{return} \PY{n}{result}
            \PY{l+s+sd}{\PYZdq{}\PYZdq{}\PYZdq{}Return a list containing the length of each word in}
        \PY{l+s+sd}{    sentence.}
        \PY{l+s+sd}{    }
        \PY{l+s+sd}{    \PYZgt{}\PYZgt{}\PYZgt{} word\PYZus{}lengths(\PYZdq{}welcome to APS360!\PYZdq{})}
        \PY{l+s+sd}{    [7, 2, 7]}
        \PY{l+s+sd}{    \PYZgt{}\PYZgt{}\PYZgt{} word\PYZus{}lengths(\PYZdq{}machine learning is so cool\PYZdq{})}
        \PY{l+s+sd}{    [7, 8, 2, 2, 4]}
        \PY{l+s+sd}{    \PYZdq{}\PYZdq{}\PYZdq{}}
        \PY{n}{word\PYZus{}lengths}\PY{p}{(}\PY{l+s+s2}{\PYZdq{}}\PY{l+s+s2}{welcome to APS360!}\PY{l+s+s2}{\PYZdq{}}\PY{p}{)}
\end{Verbatim}


\begin{Verbatim}[commandchars=\\\{\}]
{\color{outcolor}Out[{\color{outcolor}8}]:} [7, 2, 7]
\end{Verbatim}
            
    \subsection{Part 2. NumPy Exercises {[}9
pt{]}}\label{part-2.-numpy-exercises-9-pt}

In this part of the assignment, you'll be manipulating arrays usign
NumPy. Normally, we use the shorter name \texttt{np} to represent the
package \texttt{numpy}.

    \begin{Verbatim}[commandchars=\\\{\}]
{\color{incolor}In [{\color{incolor}23}]:} \PY{k+kn}{import} \PY{n+nn}{numpy} \PY{k}{as} \PY{n+nn}{np}
\end{Verbatim}


    \subsubsection{Part (a) -\/- 1pt}\label{part-a----1pt}

Load the file \texttt{matrix.csv} into a variable called \texttt{matrix}
using the function \texttt{np.loadtxt}. Make sure that
\texttt{matrix.csv} is in \textbf{the same file folder} as this
notebook.

    \begin{Verbatim}[commandchars=\\\{\}]
{\color{incolor}In [{\color{incolor}57}]:} \PY{n}{matrix} \PY{o}{=} \PY{n}{np}\PY{o}{.}\PY{n}{loadtxt}\PY{p}{(}\PY{l+s+s2}{\PYZdq{}}\PY{l+s+s2}{matrix.csv}\PY{l+s+s2}{\PYZdq{}}\PY{p}{,} \PY{n}{delimiter} \PY{o}{=}\PY{l+s+s2}{\PYZdq{}}\PY{l+s+s2}{,}\PY{l+s+s2}{\PYZdq{}}\PY{p}{)}
\end{Verbatim}


    \begin{Verbatim}[commandchars=\\\{\}]
{\color{incolor}In [{\color{incolor}58}]:} \PY{n}{matrix}
\end{Verbatim}


\begin{Verbatim}[commandchars=\\\{\}]
{\color{outcolor}Out[{\color{outcolor}58}]:} array([[1., 2., 3.],
                [4., 5., 6.],
                [7., 8., 9.]])
\end{Verbatim}
            
    \subsubsection{Part (b) -\/- 1pt}\label{part-b----1pt}

Load the file \texttt{vector.npy} into a variable called \texttt{vector}
using the function \texttt{np.load}. As before, make sure that
\texttt{vector.npy} is in \textbf{the same file folder} as this
notebook.

    \begin{Verbatim}[commandchars=\\\{\}]
{\color{incolor}In [{\color{incolor}70}]:} \PY{n}{vector} \PY{o}{=} \PY{n}{np}\PY{o}{.}\PY{n}{load}\PY{p}{(}\PY{l+s+s2}{\PYZdq{}}\PY{l+s+s2}{vector.npy}\PY{l+s+s2}{\PYZdq{}}\PY{p}{)}
         \PY{n}{vector} \PY{o}{=} \PY{n}{vector}\PY{o}{.}\PY{n}{astype}\PY{p}{(}\PY{n}{np}\PY{o}{.}\PY{n}{float}\PY{p}{)}
\end{Verbatim}


    \begin{Verbatim}[commandchars=\\\{\}]
{\color{incolor}In [{\color{incolor}71}]:} \PY{n}{vector}
\end{Verbatim}


\begin{Verbatim}[commandchars=\\\{\}]
{\color{outcolor}Out[{\color{outcolor}71}]:} array([[10.],
                [20.],
                [15.]])
\end{Verbatim}
            
    \subsubsection{Part (c) -\/- 3pt}\label{part-c----3pt}

Perform matrix multiplication \texttt{output\ =\ matrix\ x\ vector} by
using for loops to iterate through the columns and rows. Do not use any
builtin NumPy functions.

Hint: be mindful of the dimension of output

    \begin{Verbatim}[commandchars=\\\{\}]
{\color{incolor}In [{\color{incolor}139}]:} \PY{n}{output\PYZus{}list} \PY{o}{=}\PY{p}{[}\PY{p}{]}
          \PY{k}{for} \PY{n}{i} \PY{o+ow}{in} \PY{n+nb}{range}\PY{p}{(}\PY{n+nb}{len}\PY{p}{(}\PY{n}{matrix}\PY{p}{)}\PY{p}{)}\PY{p}{:}
              \PY{n}{current\PYZus{}R} \PY{o}{=} \PY{p}{[}\PY{p}{]}
              \PY{k}{for} \PY{n}{j} \PY{o+ow}{in} \PY{n+nb}{range}\PY{p}{(}\PY{n+nb}{len}\PY{p}{(}\PY{n}{vector}\PY{p}{[}\PY{l+m+mi}{0}\PY{p}{]}\PY{p}{)}\PY{p}{)}\PY{p}{:}
                  \PY{n}{current\PYZus{}sum} \PY{o}{=} \PY{l+m+mi}{0}
                  \PY{k}{for} \PY{n}{k} \PY{o+ow}{in} \PY{n+nb}{range}\PY{p}{(}\PY{n+nb}{len}\PY{p}{(}\PY{n}{vector}\PY{p}{)}\PY{p}{)}\PY{p}{:}
                      \PY{n}{current\PYZus{}sum} \PY{o}{+}\PY{o}{=} \PY{n}{matrix}\PY{p}{[}\PY{n}{i}\PY{p}{]}\PY{p}{[}\PY{n}{k}\PY{p}{]}\PY{o}{*}\PY{n}{vector}\PY{p}{[}\PY{n}{k}\PY{p}{]}\PY{p}{[}\PY{n}{j}\PY{p}{]}
                  \PY{n}{current\PYZus{}R}\PY{o}{.}\PY{n}{append}\PY{p}{(}\PY{n}{current\PYZus{}sum}\PY{p}{)}
              \PY{n}{output\PYZus{}list}\PY{o}{.}\PY{n}{append}\PY{p}{(}\PY{n}{current\PYZus{}R}\PY{p}{)}
          \PY{n}{output} \PY{o}{=} \PY{n}{np}\PY{o}{.}\PY{n}{array}\PY{p}{(}\PY{n}{output\PYZus{}list}\PY{p}{)}
            
\end{Verbatim}


    \begin{Verbatim}[commandchars=\\\{\}]
{\color{incolor}In [{\color{incolor}140}]:} \PY{n}{output}
\end{Verbatim}


\begin{Verbatim}[commandchars=\\\{\}]
{\color{outcolor}Out[{\color{outcolor}140}]:} array([[ 95.],
                 [230.],
                 [365.]])
\end{Verbatim}
            
    \subsubsection{Part (d) -\/- 1pt}\label{part-d----1pt}

Save the \texttt{output} variable into a csv file called
\texttt{output\_forloop.csv} using the function \texttt{numpy.savetxt}.

    \begin{Verbatim}[commandchars=\\\{\}]
{\color{incolor}In [{\color{incolor}79}]:} \PY{n}{np}\PY{o}{.}\PY{n}{savetxt}\PY{p}{(}\PY{l+s+s2}{\PYZdq{}}\PY{l+s+s2}{output\PYZus{}forloop.csv}\PY{l+s+s2}{\PYZdq{}}\PY{p}{,}\PY{n}{output}\PY{p}{)}
\end{Verbatim}


    \subsubsection{Part (e) -\/- 1pt}\label{part-e----1pt}

Perform matrix multiplication \texttt{output2\ =\ matrix\ x\ vector} by
using the function \texttt{numpy.dot}.

We will never actually write code as in part(c), not only because
\texttt{numpy.dot} is more concise and easier to read/write, but also
performance-wise \texttt{numpy.dot} is much faster (it is written in C
and highly optimized). In general, we will avoid for loops in our code.

    \begin{Verbatim}[commandchars=\\\{\}]
{\color{incolor}In [{\color{incolor}80}]:} \PY{n}{output2} \PY{o}{=} \PY{n}{np}\PY{o}{.}\PY{n}{dot}\PY{p}{(}\PY{n}{matrix}\PY{p}{,}\PY{n}{vector}\PY{p}{)}
\end{Verbatim}


    \begin{Verbatim}[commandchars=\\\{\}]
{\color{incolor}In [{\color{incolor}81}]:} \PY{n}{output2}
\end{Verbatim}


\begin{Verbatim}[commandchars=\\\{\}]
{\color{outcolor}Out[{\color{outcolor}81}]:} array([[ 95.],
                [230.],
                [365.]])
\end{Verbatim}
            
    \subsubsection{Part (f) -\/- 1pt}\label{part-f----1pt}

Save the \texttt{output2} variable into a csv file called
\texttt{output\_dot.npy} using the function \texttt{numpy.save}.

    \begin{Verbatim}[commandchars=\\\{\}]
{\color{incolor}In [{\color{incolor}98}]:} \PY{n}{np}\PY{o}{.}\PY{n}{save}\PY{p}{(}\PY{l+s+s2}{\PYZdq{}}\PY{l+s+s2}{output\PYZus{}dot.npy}\PY{l+s+s2}{\PYZdq{}}\PY{p}{,}\PY{n}{output2}\PY{p}{)}
\end{Verbatim}


    \subsubsection{Part (g) -\/- 1pt}\label{part-g----1pt}

As a way to test for consistency, show that the two outputs match.

    \begin{Verbatim}[commandchars=\\\{\}]
{\color{incolor}In [{\color{incolor}141}]:} \PY{n+nb}{print}\PY{p}{(}\PY{n}{output}\PY{p}{)}
          \PY{n+nb}{print}\PY{p}{(}\PY{n}{output2}\PY{p}{)}
\end{Verbatim}


    \begin{Verbatim}[commandchars=\\\{\}]
[[ 95.]
 [230.]
 [365.]]
[[ 95.]
 [230.]
 [365.]]

    \end{Verbatim}

    \subsection{Part 3. Callable Objects {[}12
pt{]}}\label{part-3.-callable-objects-12-pt}

    A \emph{callable object} is any object that can be called like a
function. In Python, any object whose class has a \texttt{\_\_call\_\_}
method will be callable. For example, we can define an \texttt{AddBias}
class that is initialized with a value \texttt{val}. When the object of
the Adder class is called with \texttt{input}, it will return the sum of
\texttt{val} and \texttt{input}:

    \begin{Verbatim}[commandchars=\\\{\}]
{\color{incolor}In [{\color{incolor}102}]:} \PY{k}{class} \PY{n+nc}{AddBias}\PY{p}{(}\PY{n+nb}{object}\PY{p}{)}\PY{p}{:}
              \PY{k}{def} \PY{n+nf}{\PYZus{}\PYZus{}init\PYZus{}\PYZus{}}\PY{p}{(}\PY{n+nb+bp}{self}\PY{p}{,} \PY{n}{val}\PY{p}{)}\PY{p}{:}
                  \PY{n+nb+bp}{self}\PY{o}{.}\PY{n}{val} \PY{o}{=} \PY{n}{val}
              \PY{k}{def} \PY{n+nf}{\PYZus{}\PYZus{}call\PYZus{}\PYZus{}}\PY{p}{(}\PY{n+nb+bp}{self}\PY{p}{,} \PY{n+nb}{input}\PY{p}{)}\PY{p}{:}
                  \PY{k}{return} \PY{n+nb+bp}{self}\PY{o}{.}\PY{n}{val} \PY{o}{+} \PY{n+nb}{input}
\end{Verbatim}


    \begin{Verbatim}[commandchars=\\\{\}]
{\color{incolor}In [{\color{incolor}103}]:} \PY{n}{add4} \PY{o}{=} \PY{n}{AddBias}\PY{p}{(}\PY{l+m+mi}{4}\PY{p}{)}
          \PY{n}{add4}\PY{p}{(}\PY{l+m+mi}{3}\PY{p}{)}
\end{Verbatim}


\begin{Verbatim}[commandchars=\\\{\}]
{\color{outcolor}Out[{\color{outcolor}103}]:} 7
\end{Verbatim}
            
    \begin{Verbatim}[commandchars=\\\{\}]
{\color{incolor}In [{\color{incolor}104}]:} \PY{c+c1}{\PYZsh{} AddBias works with numpy arrays as well}
          
          \PY{n}{add1} \PY{o}{=} \PY{n}{AddBias}\PY{p}{(}\PY{l+m+mi}{1}\PY{p}{)}
          \PY{n}{add1}\PY{p}{(}\PY{n}{np}\PY{o}{.}\PY{n}{array}\PY{p}{(}\PY{p}{[}\PY{l+m+mi}{3}\PY{p}{,}\PY{l+m+mi}{4}\PY{p}{,}\PY{l+m+mi}{5}\PY{p}{]}\PY{p}{)}\PY{p}{)}
\end{Verbatim}


\begin{Verbatim}[commandchars=\\\{\}]
{\color{outcolor}Out[{\color{outcolor}104}]:} array([4, 5, 6])
\end{Verbatim}
            
    \subsubsection{Part (a) -\/- 2pt}\label{part-a----2pt}

Create a callable object class \texttt{ElementwiseMultiply} that is
initialized with \texttt{weight}, which is a numpy array (with
1-dimension). When called on \texttt{input} of \textbf{the same shape}
as \texttt{weight}, the object will output an elementwise product of
\texttt{input} and \texttt{weight}. For example, the 1st element in the
output will be a product of the first element of \texttt{input} and
first element of \texttt{weight}. If the \texttt{input} and
\texttt{weight} have different shape, do not return anything.

    \begin{Verbatim}[commandchars=\\\{\}]
{\color{incolor}In [{\color{incolor}175}]:} \PY{k}{class} \PY{n+nc}{ElementwiseMultiply}\PY{p}{(}\PY{n+nb}{object}\PY{p}{)}\PY{p}{:}
              \PY{k}{def} \PY{n+nf}{\PYZus{}\PYZus{}init\PYZus{}\PYZus{}}\PY{p}{(}\PY{n+nb+bp}{self}\PY{p}{,}\PY{n}{weight}\PY{p}{)}\PY{p}{:}
                  \PY{n+nb+bp}{self}\PY{o}{.}\PY{n}{weight} \PY{o}{=} \PY{n}{weight}
              \PY{k}{def} \PY{n+nf}{\PYZus{}\PYZus{}call\PYZus{}\PYZus{}}\PY{p}{(}\PY{n+nb+bp}{self}\PY{p}{,} \PY{n+nb}{input}\PY{p}{)}\PY{p}{:}
                  \PY{k}{if}\PY{p}{(}\PY{n+nb}{len}\PY{p}{(}\PY{n+nb}{input}\PY{p}{)}\PY{o}{==}\PY{n+nb}{len}\PY{p}{(}\PY{n+nb+bp}{self}\PY{o}{.}\PY{n}{weight}\PY{p}{)}\PY{p}{)}\PY{p}{:}
                      \PY{k}{return} \PY{n}{np}\PY{o}{.}\PY{n}{multiply}\PY{p}{(}\PY{n+nb}{input}\PY{p}{,} \PY{n+nb+bp}{self}\PY{o}{.}\PY{n}{weight}\PY{p}{)}
              
\end{Verbatim}


    \subsubsection{Part (b) -\/- 4pt}\label{part-b----4pt}

Create a callable object class \texttt{LeakyRelu} that is initialized
with \texttt{alpha}, which is a scalar value. When called on
\texttt{input}, which may be a NumPy array, the object will output:

\begin{itemize}
\tightlist
\item
  \(f(x) = x\) if \(x \ge 0\)
\item
  \(f(x) = \alpha x\) if \(x < 0\)
\end{itemize}

To obtain full marks, do \textbf{not} use any for-loops to implement
this class.

    \begin{Verbatim}[commandchars=\\\{\}]
{\color{incolor}In [{\color{incolor}144}]:} \PY{k}{class} \PY{n+nc}{LeakyRelu}\PY{p}{(}\PY{n+nb}{object}\PY{p}{)}\PY{p}{:}
              \PY{k}{def} \PY{n+nf}{\PYZus{}\PYZus{}init\PYZus{}\PYZus{}}\PY{p}{(}\PY{n+nb+bp}{self}\PY{p}{,}\PY{n}{alpha}\PY{p}{)}\PY{p}{:}
                  \PY{n+nb+bp}{self}\PY{o}{.}\PY{n}{alpha} \PY{o}{=} \PY{n}{alpha}
              \PY{k}{def} \PY{n+nf}{\PYZus{}\PYZus{}call\PYZus{}\PYZus{}}\PY{p}{(}\PY{n+nb+bp}{self}\PY{p}{,} \PY{n+nb}{input}\PY{p}{)}\PY{p}{:}
                  \PY{n}{input\PYZus{}a} \PY{o}{=} \PY{n}{np}\PY{o}{.}\PY{n}{array}\PY{p}{(}\PY{n+nb}{input}\PY{p}{,} \PY{n}{dtype} \PY{o}{=} \PY{l+s+s1}{\PYZsq{}}\PY{l+s+s1}{float64}\PY{l+s+s1}{\PYZsq{}}\PY{p}{)}
                  \PY{n}{input\PYZus{}a}\PY{p}{[}\PY{n}{input\PYZus{}a}\PY{o}{\PYZlt{}}\PY{l+m+mi}{0}\PY{p}{]}\PY{o}{*}\PY{o}{=}\PY{n+nb+bp}{self}\PY{o}{.}\PY{n}{alpha}
                  \PY{k}{return} \PY{n}{input\PYZus{}a}
\end{Verbatim}


    \subsubsection{Part (c) -\/- 4pt}\label{part-c----4pt}

Create a callable object class \texttt{Compose} that is initialized with
\texttt{layers}, which is a list of callable objects each taking in one
argument when called. For example, \texttt{layers} can be something like
\texttt{{[}add1,\ add4{]}} that we created above. Each \texttt{add1} and
\texttt{add4} take in one argument. When \texttt{Compose} object is
called on \textbf{a single argument}, the object will output a
composition of object calls in \texttt{layers}, in the order given in
\texttt{layers} (e.g. \texttt{add1} will be called first and then
\texttt{add4} will be called after using the result from \texttt{add1}
call)

    \begin{Verbatim}[commandchars=\\\{\}]
{\color{incolor}In [{\color{incolor}145}]:} \PY{k}{class} \PY{n+nc}{Compose}\PY{p}{(}\PY{n+nb}{object}\PY{p}{)}\PY{p}{:}
              \PY{k}{def} \PY{n+nf}{\PYZus{}\PYZus{}init\PYZus{}\PYZus{}}\PY{p}{(}\PY{n+nb+bp}{self}\PY{p}{,} \PY{n}{layers}\PY{o}{=}\PY{p}{[}\PY{p}{]}\PY{p}{)}\PY{p}{:}
                  \PY{n+nb+bp}{self}\PY{o}{.}\PY{n}{layers} \PY{o}{=} \PY{n}{layers}
              \PY{k}{def} \PY{n+nf}{\PYZus{}\PYZus{}call\PYZus{}\PYZus{}}\PY{p}{(}\PY{n+nb+bp}{self}\PY{p}{,} \PY{n+nb}{input}\PY{p}{)}\PY{p}{:}
                  \PY{n}{cur\PYZus{}input} \PY{o}{=} \PY{n+nb}{input}
                  \PY{k}{for} \PY{n}{s} \PY{o+ow}{in} \PY{n+nb+bp}{self}\PY{o}{.}\PY{n}{layers}\PY{p}{:}
                      \PY{n}{cur\PYZus{}result} \PY{o}{=} \PY{n}{s}\PY{p}{(}\PY{n}{cur\PYZus{}input}\PY{p}{)}
                      \PY{n}{cur\PYZus{}input} \PY{o}{=}  \PY{n}{cur\PYZus{}result}
                  \PY{k}{return} \PY{n}{cur\PYZus{}input}
                  
\end{Verbatim}


    \subsubsection{Part (d) -\/- 2pt}\label{part-d----2pt}

Run the below code and include the output in your report.

    \begin{Verbatim}[commandchars=\\\{\}]
{\color{incolor}In [{\color{incolor}176}]:} \PY{n}{weight\PYZus{}1} \PY{o}{=} \PY{n}{np}\PY{o}{.}\PY{n}{array}\PY{p}{(}\PY{p}{[}\PY{l+m+mi}{1}\PY{p}{,} \PY{l+m+mi}{2}\PY{p}{,} \PY{l+m+mi}{3}\PY{p}{,} \PY{l+m+mi}{4}\PY{p}{]}\PY{p}{)}
          \PY{n}{weight\PYZus{}2} \PY{o}{=} \PY{n}{np}\PY{o}{.}\PY{n}{array}\PY{p}{(}\PY{p}{[}\PY{o}{\PYZhy{}}\PY{l+m+mi}{1}\PY{p}{,} \PY{o}{\PYZhy{}}\PY{l+m+mi}{2}\PY{p}{,} \PY{o}{\PYZhy{}}\PY{l+m+mi}{3}\PY{p}{,} \PY{o}{\PYZhy{}}\PY{l+m+mi}{4}\PY{p}{]}\PY{p}{)}
          \PY{n}{bias\PYZus{}1} \PY{o}{=} \PY{l+m+mi}{3}
          \PY{n}{bias\PYZus{}2} \PY{o}{=} \PY{o}{\PYZhy{}}\PY{l+m+mi}{2}
          \PY{n}{alpha} \PY{o}{=} \PY{l+m+mf}{0.1}
          
          \PY{n}{elem\PYZus{}mult\PYZus{}1} \PY{o}{=} \PY{n}{ElementwiseMultiply}\PY{p}{(}\PY{n}{weight\PYZus{}1}\PY{p}{)}
          \PY{n}{add\PYZus{}bias\PYZus{}1} \PY{o}{=} \PY{n}{AddBias}\PY{p}{(}\PY{n}{bias\PYZus{}1}\PY{p}{)}
          \PY{n}{leaky\PYZus{}relu} \PY{o}{=} \PY{n}{LeakyRelu}\PY{p}{(}\PY{n}{alpha}\PY{p}{)}
          \PY{n}{elem\PYZus{}mult\PYZus{}2} \PY{o}{=} \PY{n}{ElementwiseMultiply}\PY{p}{(}\PY{n}{weight\PYZus{}2}\PY{p}{)}
          \PY{n}{add\PYZus{}bias\PYZus{}2} \PY{o}{=} \PY{n}{AddBias}\PY{p}{(}\PY{n}{bias\PYZus{}2}\PY{p}{)}
          \PY{n}{layers} \PY{o}{=} \PY{n}{Compose}\PY{p}{(}\PY{p}{[}\PY{n}{elem\PYZus{}mult\PYZus{}1}\PY{p}{,} 
                            \PY{n}{add\PYZus{}bias\PYZus{}1}\PY{p}{,} 
                            \PY{n}{leaky\PYZus{}relu}\PY{p}{,}
                            \PY{n}{elem\PYZus{}mult\PYZus{}2}\PY{p}{,} 
                            \PY{n}{add\PYZus{}bias\PYZus{}2}\PY{p}{,} 
                            \PY{n}{leaky\PYZus{}relu}\PY{p}{]}\PY{p}{)}
          
          \PY{n+nb}{input} \PY{o}{=} \PY{n}{np}\PY{o}{.}\PY{n}{array}\PY{p}{(}\PY{p}{[}\PY{l+m+mi}{10}\PY{p}{,} \PY{l+m+mi}{5}\PY{p}{,} \PY{o}{\PYZhy{}}\PY{l+m+mi}{5}\PY{p}{,} \PY{o}{\PYZhy{}}\PY{l+m+mi}{10}\PY{p}{]}\PY{p}{)}
          \PY{n+nb}{print}\PY{p}{(}\PY{l+s+s2}{\PYZdq{}}\PY{l+s+s2}{Input: }\PY{l+s+s2}{\PYZdq{}}\PY{p}{,} \PY{n+nb}{input}\PY{p}{)}
          
          \PY{n}{output} \PY{o}{=} \PY{n}{layers}\PY{p}{(}\PY{n+nb}{input}\PY{p}{)}
          \PY{n+nb}{print}\PY{p}{(}\PY{l+s+s2}{\PYZdq{}}\PY{l+s+s2}{Output:}\PY{l+s+s2}{\PYZdq{}}\PY{p}{,} \PY{n}{output}\PY{p}{)}
\end{Verbatim}


    \begin{Verbatim}[commandchars=\\\{\}]
Input:  [ 10   5  -5 -10]
Output: [-1.5 -2.8  1.6 12.8]

    \end{Verbatim}

    \subsection{Part 4. Images {[}13 pt{]}}\label{part-4.-images-13-pt}

A picture or image can be represented as a NumPy array of ``pixels'',
with dimensions H × W × C, where H is the height of the image, W is the
width of the image, and C is the number of colour channels. Typically we
will use an image with channels that give the the Red, Green, and Blue
``level'' of each pixel, which is referred to with the short form RGB.

You will write Python code to load an image, and perform several array
manipulations to the image and visualize their effects. You'll need the
file \texttt{dog\_mochi.png} from the same place you downloaded this
assignment. Save the output images in the same directory as the Jupyter
Notebook.

    \begin{Verbatim}[commandchars=\\\{\}]
{\color{incolor}In [{\color{incolor}105}]:} \PY{k+kn}{import} \PY{n+nn}{matplotlib}\PY{n+nn}{.}\PY{n+nn}{pyplot} \PY{k}{as} \PY{n+nn}{plt}
\end{Verbatim}


    \subsubsection{Part (a) -\/- 1 pt}\label{part-a----1-pt}

Load the image \texttt{dog\_mochi.png} into the variable \texttt{img}
using the \texttt{pyplot.imread} function. This is a photograph of a dog
whose name is Mochi.

    \begin{Verbatim}[commandchars=\\\{\}]
{\color{incolor}In [{\color{incolor}156}]:} \PY{n}{img} \PY{o}{=} \PY{n}{plt}\PY{o}{.}\PY{n}{imread}\PY{p}{(}\PY{l+s+s2}{\PYZdq{}}\PY{l+s+s2}{dog\PYZus{}mochi.png}\PY{l+s+s2}{\PYZdq{}}\PY{p}{)}
\end{Verbatim}


    \subsubsection{Part (b) -\/- 1pt}\label{part-b----1pt}

Use the function \texttt{plt.imshow} to visualize \texttt{img}.

This function will also show the coordinate system used to identify
pixels. The origin is at the top left corner, and the first dimension
indicates the Y (row) direction, and the second dimension indicates the
X (column) dimension.

    \begin{Verbatim}[commandchars=\\\{\}]
{\color{incolor}In [{\color{incolor}157}]:} \PY{n}{plt}\PY{o}{.}\PY{n}{imshow}\PY{p}{(}\PY{n}{img}\PY{p}{)}
\end{Verbatim}


\begin{Verbatim}[commandchars=\\\{\}]
{\color{outcolor}Out[{\color{outcolor}157}]:} <matplotlib.image.AxesImage at 0x1a5f4454128>
\end{Verbatim}
            
    \begin{center}
    \adjustimage{max size={0.9\linewidth}{0.9\paperheight}}{output_46_1.png}
    \end{center}
    { \hspace*{\fill} \\}
    
    \subsubsection{Part (c) -\/- 2pt}\label{part-c----2pt}

What is the pixel coordinate of Mochi's eye? Show the value of each of
the 3 channels on a pixel cooresponding to Mochi's eye.

The value for each channel in the original image ranges from 0 (darkest)
to 255 (lightest). However, when loading an image through Matplotlib,
this range will be scaled to be from 0 (darkest) to 1 (brightest)
instead, and will be a real number, rather than an integer.

    \begin{Verbatim}[commandchars=\\\{\}]
{\color{incolor}In [{\color{incolor}158}]:} \PY{n}{img}\PY{p}{[}\PY{l+m+mi}{100}\PY{p}{,}\PY{l+m+mi}{50}\PY{p}{]}
\end{Verbatim}


\begin{Verbatim}[commandchars=\\\{\}]
{\color{outcolor}Out[{\color{outcolor}158}]:} array([0.58431375, 0.41568628, 0.2509804 , 1.        ], dtype=float32)
\end{Verbatim}
            
    \subsubsection{Part (d) -\/- 2pt}\label{part-d----2pt}

Modify the image by adding a constant value of 0.25 to each pixel in the
\texttt{img} and store the result in the variable \texttt{img\_add}.
Note that, since the range for the pixels needs to be between {[}0,
1{]}, you will also need to clip img\_add to be in the range {[}0, 1{]}
using \texttt{numpy.clip}. Clipping sets any value that is outside of
the desired range to the closest endpoint. Display the image using
\texttt{plt.imshow}.

    \begin{Verbatim}[commandchars=\\\{\}]
{\color{incolor}In [{\color{incolor}162}]:} \PY{n}{img\PYZus{}add} \PY{o}{=} \PY{n}{img}\PY{o}{+}\PY{l+m+mf}{0.25}
          \PY{n}{np}\PY{o}{.}\PY{n}{clip}\PY{p}{(}\PY{n}{img\PYZus{}add}\PY{p}{,} \PY{l+m+mi}{0}\PY{p}{,} \PY{l+m+mi}{1}\PY{p}{,} \PY{n}{out} \PY{o}{=} \PY{n}{img\PYZus{}add}\PY{p}{)}
          \PY{n}{plt}\PY{o}{.}\PY{n}{imshow}\PY{p}{(}\PY{n}{img\PYZus{}add}\PY{p}{)}
\end{Verbatim}


\begin{Verbatim}[commandchars=\\\{\}]
{\color{outcolor}Out[{\color{outcolor}162}]:} <matplotlib.image.AxesImage at 0x1a5f4cdb160>
\end{Verbatim}
            
    \begin{center}
    \adjustimage{max size={0.9\linewidth}{0.9\paperheight}}{output_50_1.png}
    \end{center}
    { \hspace*{\fill} \\}
    
    \subsubsection{Part (e) -\/- 3pt}\label{part-e----3pt}

From the original image, create three images that separate out the three
colour channels (red, green and blue).

Hint: First create an array initialized with zeros, then copy over the
specific channel's 2D content from img.

    \begin{Verbatim}[commandchars=\\\{\}]
{\color{incolor}In [{\color{incolor}166}]:} \PY{n}{red\PYZus{}im} \PY{o}{=} \PY{n}{img}\PY{p}{[}\PY{p}{:}\PY{p}{,} \PY{p}{:}\PY{p}{,} \PY{l+m+mi}{0}\PY{p}{]}
          \PY{n}{blu\PYZus{}im} \PY{o}{=} \PY{n}{img}\PY{p}{[}\PY{p}{:}\PY{p}{,}\PY{p}{:}\PY{p}{,}\PY{l+m+mi}{1}\PY{p}{]}
          \PY{n}{gren\PYZus{}im} \PY{o}{=} \PY{n}{img}\PY{p}{[}\PY{p}{:}\PY{p}{,}\PY{p}{:}\PY{p}{,}\PY{l+m+mi}{2}\PY{p}{]}
          
          \PY{c+c1}{\PYZsh{}plt.imshow(red\PYZus{}im)}
          \PY{c+c1}{\PYZsh{}plt.imshow(blu\PYZus{}im)}
          \PY{c+c1}{\PYZsh{}plt.imshow(gren\PYZus{}im)}
\end{Verbatim}


\begin{Verbatim}[commandchars=\\\{\}]
{\color{outcolor}Out[{\color{outcolor}166}]:} <matplotlib.image.AxesImage at 0x1a5f6199ef0>
\end{Verbatim}
            
    \begin{center}
    \adjustimage{max size={0.9\linewidth}{0.9\paperheight}}{output_52_1.png}
    \end{center}
    { \hspace*{\fill} \\}
    
    \subsubsection{Part (f) -\/- 3pt}\label{part-f----3pt}

Crop the image to only show Mochis face. Your image should be square.
Display the image.

    \begin{Verbatim}[commandchars=\\\{\}]
{\color{incolor}In [{\color{incolor}168}]:} \PY{n}{img\PYZus{}face} \PY{o}{=} \PY{n}{img}\PY{p}{[}\PY{l+m+mi}{30}\PY{p}{:}\PY{l+m+mi}{110}\PY{p}{,}\PY{l+m+mi}{50}\PY{p}{:}\PY{l+m+mi}{130}\PY{p}{,}\PY{p}{:}\PY{p}{]}
          \PY{n}{plt}\PY{o}{.}\PY{n}{imshow}\PY{p}{(}\PY{n}{img\PYZus{}face}\PY{p}{)}
\end{Verbatim}


\begin{Verbatim}[commandchars=\\\{\}]
{\color{outcolor}Out[{\color{outcolor}168}]:} <matplotlib.image.AxesImage at 0x1a5f61f1358>
\end{Verbatim}
            
    \begin{center}
    \adjustimage{max size={0.9\linewidth}{0.9\paperheight}}{output_54_1.png}
    \end{center}
    { \hspace*{\fill} \\}
    
    \subsubsection{Part (g) -\/- 1pt}\label{part-g----1pt}

Finally, save the image from part (f) using \texttt{plt.imsave} as the
filename \texttt{dog\_name.png}.

    \begin{Verbatim}[commandchars=\\\{\}]
{\color{incolor}In [{\color{incolor}169}]:} \PY{n}{plt}\PY{o}{.}\PY{n}{imsave}\PY{p}{(}\PY{l+s+s2}{\PYZdq{}}\PY{l+s+s2}{dog\PYZus{}name.png}\PY{l+s+s2}{\PYZdq{}}\PY{p}{,}\PY{n}{img\PYZus{}face}\PY{p}{)}
\end{Verbatim}



    % Add a bibliography block to the postdoc
    
    
    
    \end{document}
